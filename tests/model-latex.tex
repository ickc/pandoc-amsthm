\theoremstyle{plain}
\newtheorem{Theorem}{Theorem}[section]
\newtheorem{Lemma}[Theorem]{Lemma}
\newtheorem{Corollary}[Theorem]{Corollary}
\newtheorem*{With Space}{With Space}
\newtheorem{Proposition}{Proposition}[section]
\newtheorem{Conjecture}[Proposition]{Conjecture}
\newtheorem{WithoutSpace}{WithoutSpace}[section]
\newtheorem{KL}{Klein’s Lemma}[section]
\theoremstyle{definition}
\newtheorem{Definition}{Definition}[section]
\theoremstyle{remark}
\newtheorem{Case}{Case}[section]

\hypertarget{demo}{%
\chapter{Demo}\label{demo}}

\begin{Theorem}[within parenthesis]
plain theoremstyle \emph{here}
\end{Theorem}

\begin{Theorem}\label{simplestEquation}
\leavevmode\vadjust pre{\hypertarget{simplestEquation}{}}%
Label and reference:

\[E=mc^2\]
\end{Theorem}

From the \ref{simplestEquation}, we see that\ldots{}

Or we can use pandoc-crossref style \ref{simplestEquation} as well.

\begin{With Space}[\textbf{This} is \emph{markdown}.]
Environment name has a space, and is unnumbered.
\end{With Space}

\begin{Lemma}[can cite \ref{simplestEquation}]
This one share counter with Theorem.

Cite inside info only works with \texttt{ref\{...\}} syntax. As the
conversion using \texttt{{[}@...{]}} to AST and walk and back to LaTeX
would be too complex.
\end{Lemma}

\begin{Definition}[within parenthesis]
definition theoremstyle here
\end{Definition}

\begin{Case}[within parenthesis]
remark theoremstyle here
\end{Case}

\begin{proof}[Proof of the Main Theorem]
Predefined proof theoremstyle here
\end{proof}

\begin{KL}
Klein's Lemma from amsthm doc.
\end{KL}

\begin{Definition}
\begin{verbatim}
code here
\end{verbatim}
\end{Definition}

\hypertarget{counter-test}{%
\chapter{Counter test}\label{counter-test}}

\begin{Theorem}
some theorem
\end{Theorem}

\begin{Theorem}
some theorem
\end{Theorem}

\hypertarget{next-level}{%
\section{Next level}\label{next-level}}

\begin{Theorem}
some theorem
\end{Theorem}

\begin{Theorem}
some theorem
\end{Theorem}

\hypertarget{level-3}{%
\subsection{Level 3}\label{level-3}}

\begin{Theorem}
some theorem
\end{Theorem}

\begin{Theorem}
some theorem
\end{Theorem}

\hypertarget{level-4}{%
\subsubsection{Level 4}\label{level-4}}

\begin{Theorem}
some theorem
\end{Theorem}

\begin{Theorem}
some theorem
\end{Theorem}

\hypertarget{level-5}{%
\paragraph{Level 5}\label{level-5}}

\begin{Theorem}
some theorem
\end{Theorem}

\begin{Theorem}
some theorem
\end{Theorem}

\hypertarget{level-6}{%
\subparagraph{Level 6}\label{level-6}}

\begin{Theorem}
some theorem
\end{Theorem}

\begin{Theorem}
some theorem
\end{Theorem}
