\documentclass[12pt,english,letterpaper,oneside, article]{memoir}
\usepackage[]{lmodern}
\usepackage{amssymb,amsmath}
\usepackage{ifxetex,ifluatex}
\usepackage{fixltx2e} % provides \textsubscript
\ifnum 0\ifxetex 1\fi\ifluatex 1\fi=0 % if pdftex
  \usepackage[T1]{fontenc}
  \usepackage[utf8]{inputenc}
\else % if luatex or xelatex
  \ifxetex
    \usepackage{mathspec}
  \else
    \usepackage{fontspec}
  \fi
  \defaultfontfeatures{Ligatures=TeX,Scale=MatchLowercase}
\fi
% use upquote if available, for straight quotes in verbatim environments
\IfFileExists{upquote.sty}{\usepackage{upquote}}{}
% use microtype if available
\IfFileExists{microtype.sty}{%
\usepackage{microtype}
\UseMicrotypeSet[protrusion]{basicmath} % disable protrusion for tt fonts
}{}
\usepackage[inner=1.5in, outer=1.5in, top=1.5in, bottom=1.75in]{geometry}
\usepackage{hyperref}
\PassOptionsToPackage{usenames,dvipsnames}{color} % color is loaded by hyperref
\hypersetup{unicode=true,
            pdftitle={pandoc with amsthm},
            pdfauthor={Kolen Cheung},
            pdfkeywords={pandoc, amsthm, LaTeX},
            colorlinks=true,
            linkcolor=blue,
            citecolor=blue,
            urlcolor=blue,
            breaklinks=true}
\urlstyle{same}  % don't use monospace font for urls
\ifnum 0\ifxetex 1\fi\ifluatex 1\fi=0 % if pdftex
  \usepackage[shorthands=off,main=english]{babel}
\else
  \usepackage{polyglossia}
  \setmainlanguage[]{english}
\fi
\IfFileExists{parskip.sty}{%
\usepackage{parskip}
}{% else
\setlength{\parindent}{0pt}
\setlength{\parskip}{6pt plus 2pt minus 1pt}
}
\setlength{\emergencystretch}{3em}  % prevent overfull lines
\providecommand{\tightlist}{%
  \setlength{\itemsep}{0pt}\setlength{\parskip}{0pt}}
\setcounter{secnumdepth}{5}
% Redefines (sub)paragraphs to behave more like sections
\ifx\paragraph\undefined\else
\let\oldparagraph\paragraph
\renewcommand{\paragraph}[1]{\oldparagraph{#1}\mbox{}}
\fi
\ifx\subparagraph\undefined\else
\let\oldsubparagraph\subparagraph
\renewcommand{\subparagraph}[1]{\oldsubparagraph{#1}\mbox{}}
\fi
\usepackage{amsthm}

\theoremstyle{plain} % default
\newtheorem{theorem}{Theorem}[chapter]
\newtheorem{lemma}[theorem]{Lemma}
\newtheorem{proposition}[theorem]{Proposition}
\newtheorem*{corollary}{Corollary}

\theoremstyle{definition}
\newtheorem{definition}{Definition}[chapter]
\newtheorem{conjecture}{Conjecture}[chapter]
\newtheorem{example}{Example}[chapter]
\newtheorem{postulate}{Postulate}[chapter]
\newtheorem{problem}{Problem}[chapter]

\theoremstyle{remark}
\newtheorem*{remark}{Remark}
\newtheorem*{note}{Note}
\newtheorem{case}{Case}

% proof is predefined, see documentation

\title{pandoc with amsthm}
\author{Kolen Cheung}
\date{\today}

\begin{document}
\maketitle

{
\hypersetup{linkcolor=blue}
\setcounter{tocdepth}{5}
\tableofcontents
}
\chapter{Theorem}\label{theorem}

\begin{theorem}

\[\nabla \times \mathbf{E} = - \frac{\partial \mathbf{B}}{\partial t}\]

\end{theorem}

\begin{lemma}

\[\nabla \times \mathbf{E} = - \frac{\partial \mathbf{B}}{\partial t}\]

\end{lemma}

\begin{proposition}

\[\nabla \times \mathbf{E} = - \frac{\partial \mathbf{B}}{\partial t}\]

\end{proposition}

\begin{corollary}

\[\nabla \times \mathbf{E} = - \frac{\partial \mathbf{B}}{\partial t}\]

\end{corollary}

\begin{definition}

\[E=mc^2\]

\end{definition}

\begin{conjecture}

\[E=mc^2\]

\end{conjecture}

\begin{example}

\[E=mc^2\]

\end{example}

\begin{postulate}

\[E=mc^2\]

\end{postulate}

\begin{problem}

\[E=mc^2\]

\end{problem}

\begin{remark}

\[\nabla \times \mathbf{E} = - \frac{\partial \mathbf{B}}{\partial t}\]

\end{remark}

\begin{note}

\[\nabla \times \mathbf{E} = - \frac{\partial \mathbf{B}}{\partial t}\]

\end{note}

\begin{case}

\[\nabla \times \mathbf{E} = - \frac{\partial \mathbf{B}}{\partial t}\]

\end{case}

\begin{proof}

\[E=mc^2\]

\end{proof}

\begin{case}

\[\nabla \times \mathbf{E} = - \frac{\partial \mathbf{B}}{\partial t}\]

\end{case}

\end{document}
