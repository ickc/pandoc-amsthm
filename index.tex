\documentclass[english,oneside, article]{memoir}
\usepackage{lmodern}
\usepackage{amssymb,amsmath}
\usepackage{amsthm}
\theoremstyle{plain}
\newtheorem{Theorem}{Theorem}[chapter]
\newtheorem*{Lemma}{Lemma}
\newtheorem*{Proposition}{Proposition}
\newtheorem*{Corollary}{Corollary}
\theoremstyle{definition}
\newtheorem{Definition}{Definition}[chapter]
\newtheorem{Conjecture}{Conjecture}[chapter]
\newtheorem{Example}{Example}[chapter]
\newtheorem{Postulate}{Postulate}[chapter]
\newtheorem{Problem}{Problem}[chapter]
\theoremstyle{remark}
\newtheorem{Case}{Case}[chapter]
\newtheorem*{Remark}{Remark}
\newtheorem*{Note}{Note}
\usepackage{ifxetex,ifluatex}
\usepackage{fixltx2e} % provides \textsubscript
\ifnum 0\ifxetex 1\fi\ifluatex 1\fi=0 % if pdftex
  \usepackage[T1]{fontenc}
  \usepackage[utf8]{inputenc}
\else % if luatex or xelatex
  \ifxetex
    \usepackage{mathspec}
  \else
    \usepackage{fontspec}
  \fi
  \defaultfontfeatures{Ligatures=TeX,Scale=MatchLowercase}
\fi
% use upquote if available, for straight quotes in verbatim environments
\IfFileExists{upquote.sty}{\usepackage{upquote}}{}
% use microtype if available
\IfFileExists{microtype.sty}{%
\usepackage{microtype}
\UseMicrotypeSet[protrusion]{basicmath} % disable protrusion for tt fonts
}{}
\usepackage{hyperref}
\PassOptionsToPackage{usenames,dvipsnames}{color} % color is loaded by hyperref
\hypersetup{unicode=true,
            pdftitle={Pandoc with Amsthm Defined in YAML Front Matter},
            pdfauthor={Kolen Cheung},
            pdfkeywords={pandoc, amsthm, LaTeX, yaml},
            colorlinks=true,
            linkcolor=Maroon,
            citecolor=Blue,
            urlcolor=Blue,
            breaklinks=true}
\urlstyle{same}  % don't use monospace font for urls
\ifnum 0\ifxetex 1\fi\ifluatex 1\fi=0 % if pdftex
  \usepackage[shorthands=off,main=english]{babel}
\else
  \usepackage{polyglossia}
  \setmainlanguage[]{english}
\fi
\IfFileExists{parskip.sty}{%
\usepackage{parskip}
}{% else
\setlength{\parindent}{0pt}
\setlength{\parskip}{6pt plus 2pt minus 1pt}
}
\setlength{\emergencystretch}{3em}  % prevent overfull lines
\providecommand{\tightlist}{%
  \setlength{\itemsep}{0pt}\setlength{\parskip}{0pt}}
\setcounter{secnumdepth}{0}
% Redefines (sub)paragraphs to behave more like sections
\ifx\paragraph\undefined\else
\let\oldparagraph\paragraph
\renewcommand{\paragraph}[1]{\oldparagraph{#1}\mbox{}}
\fi
\ifx\subparagraph\undefined\else
\let\oldsubparagraph\subparagraph
\renewcommand{\subparagraph}[1]{\oldsubparagraph{#1}\mbox{}}
\fi

\title{Pandoc with Amsthm Defined in YAML Front Matter}
\author{Kolen Cheung}
\date{\today}

\begin{document}
\maketitle

{
\hypersetup{linkcolor=black}
\setcounter{tocdepth}{5}
\tableofcontents
}
\chapter{First Heading}\label{first-heading}

\begin{Theorem}

The Riemann zeta function is defined for complex \(s\) with real part
greater than \(1\) by the absolutely convergent infinite series

\[\zeta(s) = \sum_{n=1}^\infty \frac{1}{n^s} = \frac{1}{1^s} + \frac{1}{2^s} + \frac{1}{3^s} + \cdots.\]

\end{Theorem}

\begin{Lemma}

The Riemann zeta function is defined for complex \(s\) with real part
greater than \(1\) by the absolutely convergent infinite series

\[\zeta(s) = \sum_{n=1}^\infty \frac{1}{n^s} = \frac{1}{1^s} + \frac{1}{2^s} + \frac{1}{3^s} + \cdots.\]

\end{Lemma}

\begin{Proposition}

The Riemann zeta function is defined for complex \(s\) with real part
greater than \(1\) by the absolutely convergent infinite series

\[\zeta(s) = \sum_{n=1}^\infty \frac{1}{n^s} = \frac{1}{1^s} + \frac{1}{2^s} + \frac{1}{3^s} + \cdots.\]

\end{Proposition}

\begin{Corollary}

The Riemann zeta function is defined for complex \(s\) with real part
greater than \(1\) by the absolutely convergent infinite series

\[\zeta(s) = \sum_{n=1}^\infty \frac{1}{n^s} = \frac{1}{1^s} + \frac{1}{2^s} + \frac{1}{3^s} + \cdots.\]

\end{Corollary}

\begin{Definition}

Leonhard Euler showed that this series equals the Euler product

\[\zeta(s) = \prod_{p \text{ prime}} \frac{1}{1-p^{-s}}= \frac{1}{1-2^{-s}}\cdot\frac{1}{1-3^{-s}}\cdot\frac{1}{1-5^{-s}}\cdot\frac{1}{1-7^{-s}} \cdots \frac{1}{1-p^{-s}} \cdots\]

\end{Definition}

\begin{Conjecture}

Leonhard Euler showed that this series equals the Euler product

\[\zeta(s) = \prod_{p \text{ prime}} \frac{1}{1-p^{-s}}= \frac{1}{1-2^{-s}}\cdot\frac{1}{1-3^{-s}}\cdot\frac{1}{1-5^{-s}}\cdot\frac{1}{1-7^{-s}} \cdots \frac{1}{1-p^{-s}} \cdots\]

\end{Conjecture}

\begin{Example}

Leonhard Euler showed that this series equals the Euler product

\[\zeta(s) = \prod_{p \text{ prime}} \frac{1}{1-p^{-s}}= \frac{1}{1-2^{-s}}\cdot\frac{1}{1-3^{-s}}\cdot\frac{1}{1-5^{-s}}\cdot\frac{1}{1-7^{-s}} \cdots \frac{1}{1-p^{-s}} \cdots\]

\end{Example}

\begin{Postulate}

Leonhard Euler showed that this series equals the Euler product

\[\zeta(s) = \prod_{p \text{ prime}} \frac{1}{1-p^{-s}}= \frac{1}{1-2^{-s}}\cdot\frac{1}{1-3^{-s}}\cdot\frac{1}{1-5^{-s}}\cdot\frac{1}{1-7^{-s}} \cdots \frac{1}{1-p^{-s}} \cdots\]

\end{Postulate}

\begin{Problem}

Leonhard Euler showed that this series equals the Euler product

\[\zeta(s) = \prod_{p \text{ prime}} \frac{1}{1-p^{-s}}= \frac{1}{1-2^{-s}}\cdot\frac{1}{1-3^{-s}}\cdot\frac{1}{1-5^{-s}}\cdot\frac{1}{1-7^{-s}} \cdots \frac{1}{1-p^{-s}} \cdots\]

\end{Problem}

\begin{Remark}

The Riemann zeta function is defined for complex \(s\) with real part
greater than \(1\) by the absolutely convergent infinite series

\[\zeta(s) = \sum_{n=1}^\infty \frac{1}{n^s} = \frac{1}{1^s} + \frac{1}{2^s} + \frac{1}{3^s} + \cdots.\]

\end{Remark}

\begin{Note}

The Riemann zeta function is defined for complex \(s\) with real part
greater than \(1\) by the absolutely convergent infinite series

\[\zeta(s) = \sum_{n=1}^\infty \frac{1}{n^s} = \frac{1}{1^s} + \frac{1}{2^s} + \frac{1}{3^s} + \cdots.\]

\end{Note}

\begin{Case}

The Riemann zeta function is defined for complex \(s\) with real part
greater than \(1\) by the absolutely convergent infinite series

\[\zeta(s) = \sum_{n=1}^\infty \frac{1}{n^s} = \frac{1}{1^s} + \frac{1}{2^s} + \frac{1}{3^s} + \cdots.\]

\end{Case}

\begin{proof}

Leonhard Euler showed that this series equals the Euler product

\[\zeta(s) = \prod_{p \text{ prime}} \frac{1}{1-p^{-s}}= \frac{1}{1-2^{-s}}\cdot\frac{1}{1-3^{-s}}\cdot\frac{1}{1-5^{-s}}\cdot\frac{1}{1-7^{-s}} \cdots \frac{1}{1-p^{-s}} \cdots\]

\end{proof}

\textbf{Repeating once:}

\begin{Theorem}

The Riemann zeta function is defined for complex \(s\) with real part
greater than \(1\) by the absolutely convergent infinite series

\[\zeta(s) = \sum_{n=1}^\infty \frac{1}{n^s} = \frac{1}{1^s} + \frac{1}{2^s} + \frac{1}{3^s} + \cdots.\]

\end{Theorem}

\begin{Lemma}

The Riemann zeta function is defined for complex \(s\) with real part
greater than \(1\) by the absolutely convergent infinite series

\[\zeta(s) = \sum_{n=1}^\infty \frac{1}{n^s} = \frac{1}{1^s} + \frac{1}{2^s} + \frac{1}{3^s} + \cdots.\]

\end{Lemma}

\begin{Proposition}

The Riemann zeta function is defined for complex \(s\) with real part
greater than \(1\) by the absolutely convergent infinite series

\[\zeta(s) = \sum_{n=1}^\infty \frac{1}{n^s} = \frac{1}{1^s} + \frac{1}{2^s} + \frac{1}{3^s} + \cdots.\]

\end{Proposition}

\begin{Corollary}

The Riemann zeta function is defined for complex \(s\) with real part
greater than \(1\) by the absolutely convergent infinite series

\[\zeta(s) = \sum_{n=1}^\infty \frac{1}{n^s} = \frac{1}{1^s} + \frac{1}{2^s} + \frac{1}{3^s} + \cdots.\]

\end{Corollary}

\begin{Definition}

Leonhard Euler showed that this series equals the Euler product

\[\zeta(s) = \prod_{p \text{ prime}} \frac{1}{1-p^{-s}}= \frac{1}{1-2^{-s}}\cdot\frac{1}{1-3^{-s}}\cdot\frac{1}{1-5^{-s}}\cdot\frac{1}{1-7^{-s}} \cdots \frac{1}{1-p^{-s}} \cdots\]

\end{Definition}

\begin{Conjecture}

Leonhard Euler showed that this series equals the Euler product

\[\zeta(s) = \prod_{p \text{ prime}} \frac{1}{1-p^{-s}}= \frac{1}{1-2^{-s}}\cdot\frac{1}{1-3^{-s}}\cdot\frac{1}{1-5^{-s}}\cdot\frac{1}{1-7^{-s}} \cdots \frac{1}{1-p^{-s}} \cdots\]

\end{Conjecture}

\begin{Example}

Leonhard Euler showed that this series equals the Euler product

\[\zeta(s) = \prod_{p \text{ prime}} \frac{1}{1-p^{-s}}= \frac{1}{1-2^{-s}}\cdot\frac{1}{1-3^{-s}}\cdot\frac{1}{1-5^{-s}}\cdot\frac{1}{1-7^{-s}} \cdots \frac{1}{1-p^{-s}} \cdots\]

\end{Example}

\begin{Postulate}

Leonhard Euler showed that this series equals the Euler product

\[\zeta(s) = \prod_{p \text{ prime}} \frac{1}{1-p^{-s}}= \frac{1}{1-2^{-s}}\cdot\frac{1}{1-3^{-s}}\cdot\frac{1}{1-5^{-s}}\cdot\frac{1}{1-7^{-s}} \cdots \frac{1}{1-p^{-s}} \cdots\]

\end{Postulate}

\begin{Problem}

Leonhard Euler showed that this series equals the Euler product

\[\zeta(s) = \prod_{p \text{ prime}} \frac{1}{1-p^{-s}}= \frac{1}{1-2^{-s}}\cdot\frac{1}{1-3^{-s}}\cdot\frac{1}{1-5^{-s}}\cdot\frac{1}{1-7^{-s}} \cdots \frac{1}{1-p^{-s}} \cdots\]

\end{Problem}

\begin{Remark}

The Riemann zeta function is defined for complex \(s\) with real part
greater than \(1\) by the absolutely convergent infinite series

\[\zeta(s) = \sum_{n=1}^\infty \frac{1}{n^s} = \frac{1}{1^s} + \frac{1}{2^s} + \frac{1}{3^s} + \cdots.\]

\end{Remark}

\begin{Note}

The Riemann zeta function is defined for complex \(s\) with real part
greater than \(1\) by the absolutely convergent infinite series

\[\zeta(s) = \sum_{n=1}^\infty \frac{1}{n^s} = \frac{1}{1^s} + \frac{1}{2^s} + \frac{1}{3^s} + \cdots.\]

\end{Note}

\begin{Case}

The Riemann zeta function is defined for complex \(s\) with real part
greater than \(1\) by the absolutely convergent infinite series

\[\zeta(s) = \sum_{n=1}^\infty \frac{1}{n^s} = \frac{1}{1^s} + \frac{1}{2^s} + \frac{1}{3^s} + \cdots.\]

\end{Case}

\begin{proof}

Leonhard Euler showed that this series equals the Euler product

\[\zeta(s) = \prod_{p \text{ prime}} \frac{1}{1-p^{-s}}= \frac{1}{1-2^{-s}}\cdot\frac{1}{1-3^{-s}}\cdot\frac{1}{1-5^{-s}}\cdot\frac{1}{1-7^{-s}} \cdots \frac{1}{1-p^{-s}} \cdots\]

\end{proof}

\chapter{Second Heading}\label{second-heading}

\begin{Theorem}

The Riemann zeta function is defined for complex \(s\) with real part
greater than \(1\) by the absolutely convergent infinite series

\[\zeta(s) = \sum_{n=1}^\infty \frac{1}{n^s} = \frac{1}{1^s} + \frac{1}{2^s} + \frac{1}{3^s} + \cdots.\]

\end{Theorem}

\begin{Lemma}

The Riemann zeta function is defined for complex \(s\) with real part
greater than \(1\) by the absolutely convergent infinite series

\[\zeta(s) = \sum_{n=1}^\infty \frac{1}{n^s} = \frac{1}{1^s} + \frac{1}{2^s} + \frac{1}{3^s} + \cdots.\]

\end{Lemma}

\begin{Proposition}

The Riemann zeta function is defined for complex \(s\) with real part
greater than \(1\) by the absolutely convergent infinite series

\[\zeta(s) = \sum_{n=1}^\infty \frac{1}{n^s} = \frac{1}{1^s} + \frac{1}{2^s} + \frac{1}{3^s} + \cdots.\]

\end{Proposition}

\begin{Corollary}

The Riemann zeta function is defined for complex \(s\) with real part
greater than \(1\) by the absolutely convergent infinite series

\[\zeta(s) = \sum_{n=1}^\infty \frac{1}{n^s} = \frac{1}{1^s} + \frac{1}{2^s} + \frac{1}{3^s} + \cdots.\]

\end{Corollary}

\begin{Definition}

Leonhard Euler showed that this series equals the Euler product

\[\zeta(s) = \prod_{p \text{ prime}} \frac{1}{1-p^{-s}}= \frac{1}{1-2^{-s}}\cdot\frac{1}{1-3^{-s}}\cdot\frac{1}{1-5^{-s}}\cdot\frac{1}{1-7^{-s}} \cdots \frac{1}{1-p^{-s}} \cdots\]

\end{Definition}

\begin{Conjecture}

Leonhard Euler showed that this series equals the Euler product

\[\zeta(s) = \prod_{p \text{ prime}} \frac{1}{1-p^{-s}}= \frac{1}{1-2^{-s}}\cdot\frac{1}{1-3^{-s}}\cdot\frac{1}{1-5^{-s}}\cdot\frac{1}{1-7^{-s}} \cdots \frac{1}{1-p^{-s}} \cdots\]

\end{Conjecture}

\begin{Example}

Leonhard Euler showed that this series equals the Euler product

\[\zeta(s) = \prod_{p \text{ prime}} \frac{1}{1-p^{-s}}= \frac{1}{1-2^{-s}}\cdot\frac{1}{1-3^{-s}}\cdot\frac{1}{1-5^{-s}}\cdot\frac{1}{1-7^{-s}} \cdots \frac{1}{1-p^{-s}} \cdots\]

\end{Example}

\begin{Postulate}

Leonhard Euler showed that this series equals the Euler product

\[\zeta(s) = \prod_{p \text{ prime}} \frac{1}{1-p^{-s}}= \frac{1}{1-2^{-s}}\cdot\frac{1}{1-3^{-s}}\cdot\frac{1}{1-5^{-s}}\cdot\frac{1}{1-7^{-s}} \cdots \frac{1}{1-p^{-s}} \cdots\]

\end{Postulate}

\begin{Problem}

Leonhard Euler showed that this series equals the Euler product

\[\zeta(s) = \prod_{p \text{ prime}} \frac{1}{1-p^{-s}}= \frac{1}{1-2^{-s}}\cdot\frac{1}{1-3^{-s}}\cdot\frac{1}{1-5^{-s}}\cdot\frac{1}{1-7^{-s}} \cdots \frac{1}{1-p^{-s}} \cdots\]

\end{Problem}

\begin{Remark}

The Riemann zeta function is defined for complex \(s\) with real part
greater than \(1\) by the absolutely convergent infinite series

\[\zeta(s) = \sum_{n=1}^\infty \frac{1}{n^s} = \frac{1}{1^s} + \frac{1}{2^s} + \frac{1}{3^s} + \cdots.\]

\end{Remark}

\begin{Note}

The Riemann zeta function is defined for complex \(s\) with real part
greater than \(1\) by the absolutely convergent infinite series

\[\zeta(s) = \sum_{n=1}^\infty \frac{1}{n^s} = \frac{1}{1^s} + \frac{1}{2^s} + \frac{1}{3^s} + \cdots.\]

\end{Note}

\begin{Case}

The Riemann zeta function is defined for complex \(s\) with real part
greater than \(1\) by the absolutely convergent infinite series

\[\zeta(s) = \sum_{n=1}^\infty \frac{1}{n^s} = \frac{1}{1^s} + \frac{1}{2^s} + \frac{1}{3^s} + \cdots.\]

\end{Case}

\begin{proof}

Leonhard Euler showed that this series equals the Euler product

\[\zeta(s) = \prod_{p \text{ prime}} \frac{1}{1-p^{-s}}= \frac{1}{1-2^{-s}}\cdot\frac{1}{1-3^{-s}}\cdot\frac{1}{1-5^{-s}}\cdot\frac{1}{1-7^{-s}} \cdots \frac{1}{1-p^{-s}} \cdots\]

\end{proof}

\section{Subheading}\label{subheading}

\begin{Theorem}

The Riemann zeta function is defined for complex \(s\) with real part
greater than \(1\) by the absolutely convergent infinite series

\[\zeta(s) = \sum_{n=1}^\infty \frac{1}{n^s} = \frac{1}{1^s} + \frac{1}{2^s} + \frac{1}{3^s} + \cdots.\]

\end{Theorem}

\begin{Lemma}

The Riemann zeta function is defined for complex \(s\) with real part
greater than \(1\) by the absolutely convergent infinite series

\[\zeta(s) = \sum_{n=1}^\infty \frac{1}{n^s} = \frac{1}{1^s} + \frac{1}{2^s} + \frac{1}{3^s} + \cdots.\]

\end{Lemma}

\begin{Proposition}

The Riemann zeta function is defined for complex \(s\) with real part
greater than \(1\) by the absolutely convergent infinite series

\[\zeta(s) = \sum_{n=1}^\infty \frac{1}{n^s} = \frac{1}{1^s} + \frac{1}{2^s} + \frac{1}{3^s} + \cdots.\]

\end{Proposition}

\begin{Corollary}

The Riemann zeta function is defined for complex \(s\) with real part
greater than \(1\) by the absolutely convergent infinite series

\[\zeta(s) = \sum_{n=1}^\infty \frac{1}{n^s} = \frac{1}{1^s} + \frac{1}{2^s} + \frac{1}{3^s} + \cdots.\]

\end{Corollary}

\begin{Definition}

Leonhard Euler showed that this series equals the Euler product

\[\zeta(s) = \prod_{p \text{ prime}} \frac{1}{1-p^{-s}}= \frac{1}{1-2^{-s}}\cdot\frac{1}{1-3^{-s}}\cdot\frac{1}{1-5^{-s}}\cdot\frac{1}{1-7^{-s}} \cdots \frac{1}{1-p^{-s}} \cdots\]

\end{Definition}

\begin{Conjecture}

Leonhard Euler showed that this series equals the Euler product

\[\zeta(s) = \prod_{p \text{ prime}} \frac{1}{1-p^{-s}}= \frac{1}{1-2^{-s}}\cdot\frac{1}{1-3^{-s}}\cdot\frac{1}{1-5^{-s}}\cdot\frac{1}{1-7^{-s}} \cdots \frac{1}{1-p^{-s}} \cdots\]

\end{Conjecture}

\begin{Example}

Leonhard Euler showed that this series equals the Euler product

\[\zeta(s) = \prod_{p \text{ prime}} \frac{1}{1-p^{-s}}= \frac{1}{1-2^{-s}}\cdot\frac{1}{1-3^{-s}}\cdot\frac{1}{1-5^{-s}}\cdot\frac{1}{1-7^{-s}} \cdots \frac{1}{1-p^{-s}} \cdots\]

\end{Example}

\begin{Postulate}

Leonhard Euler showed that this series equals the Euler product

\[\zeta(s) = \prod_{p \text{ prime}} \frac{1}{1-p^{-s}}= \frac{1}{1-2^{-s}}\cdot\frac{1}{1-3^{-s}}\cdot\frac{1}{1-5^{-s}}\cdot\frac{1}{1-7^{-s}} \cdots \frac{1}{1-p^{-s}} \cdots\]

\end{Postulate}

\begin{Problem}

Leonhard Euler showed that this series equals the Euler product

\[\zeta(s) = \prod_{p \text{ prime}} \frac{1}{1-p^{-s}}= \frac{1}{1-2^{-s}}\cdot\frac{1}{1-3^{-s}}\cdot\frac{1}{1-5^{-s}}\cdot\frac{1}{1-7^{-s}} \cdots \frac{1}{1-p^{-s}} \cdots\]

\end{Problem}

\begin{Remark}

The Riemann zeta function is defined for complex \(s\) with real part
greater than \(1\) by the absolutely convergent infinite series

\[\zeta(s) = \sum_{n=1}^\infty \frac{1}{n^s} = \frac{1}{1^s} + \frac{1}{2^s} + \frac{1}{3^s} + \cdots.\]

\end{Remark}

\begin{Note}

The Riemann zeta function is defined for complex \(s\) with real part
greater than \(1\) by the absolutely convergent infinite series

\[\zeta(s) = \sum_{n=1}^\infty \frac{1}{n^s} = \frac{1}{1^s} + \frac{1}{2^s} + \frac{1}{3^s} + \cdots.\]

\end{Note}

\begin{Case}

The Riemann zeta function is defined for complex \(s\) with real part
greater than \(1\) by the absolutely convergent infinite series

\[\zeta(s) = \sum_{n=1}^\infty \frac{1}{n^s} = \frac{1}{1^s} + \frac{1}{2^s} + \frac{1}{3^s} + \cdots.\]

\end{Case}

\begin{proof}

Leonhard Euler showed that this series equals the Euler product

\[\zeta(s) = \prod_{p \text{ prime}} \frac{1}{1-p^{-s}}= \frac{1}{1-2^{-s}}\cdot\frac{1}{1-3^{-s}}\cdot\frac{1}{1-5^{-s}}\cdot\frac{1}{1-7^{-s}} \cdots \frac{1}{1-p^{-s}} \cdots\]

\end{proof}

\chapter{Test}\label{test}

\begin{proof}

This one has 2 amsthm classes, which should be disallowed. In this case
the filter will pick the first valid amsthm class to be the LaTeX
environment and ignore the rest.

\end{proof}

\begin{Theorem}

This one has multiple classes, where only 1 of them is amsthm class
,this should be valid.

\end{Theorem}

\end{document}
